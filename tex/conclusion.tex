\chapter{Conclusion}

In this paper, we reviewed the current literature on applying machine learning techniques to communcations systems; specifically equalization and carrier frequency offset.  We also discussed the value of robustness and adaptability in future communications systems requiring an investigation of meta-learning.
We demonstrated that neural networks can learn to learn to estimate channel characteristics for two tap channels.  We explored the using deep recursive neural networks to learn to learn to equalize.
We also used recursive neural networks to learn to learn to be a clock by rotating in a circle at different rates.  Lastly, we used deep neural networks to estimate the rate of carrier frequency offset and correct it.

All of this without using backpropagation when faced with a new environment.
Communications is a perfect application to explore the possibilities and limitations of the current meta-learning and machine learning frameworks.  


\section{Future Work}

In the future, we would like to expand our channel estimation networks to handle any number of channel taps and channel taps that also have imaginary parts.
We would also like to explore how we might be able to learn to estimate the channel even without a shared preamble.

We hope to explore and expand our research on equalization networks.  One first step will be investigating how adding logarithmic layers into the networks affects the performance.  We also want to expand our CFO estimation and correction to work with multi-tap channels and eventually combine all of the aspects in this paper to create one deep equalizer network.

In general, this paper was more of an exploration into how neural networks can learn to learn to communicate for equalization and CFO.  We would like to do a more thorough architecture search and longer training runs.  We also hope to to verify the ideas presented in this paper by training on real radio data instead of the simulated data used in this paper.